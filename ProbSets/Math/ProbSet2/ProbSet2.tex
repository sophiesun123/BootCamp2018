\documentclass[letterpaper,12pt]{article}
\usepackage{array}
\usepackage{threeparttable}
\usepackage{geometry}
\geometry{letterpaper,tmargin=1in,bmargin=1in,lmargin=1.25in,rmargin=1.25in}
\usepackage{fancyhdr,lastpage}
\pagestyle{fancy}
\lhead{}
\chead{}
\rhead{}
\lfoot{}
\cfoot{}
\rfoot{\footnotesize\textsl{Page \thepage\ of \pageref{LastPage}}}
\renewcommand\headrulewidth{0pt}
\renewcommand\footrulewidth{0pt}
\usepackage[format=hang,font=normalsize,labelfont=bf]{caption}
\usepackage{listings}
\lstset{frame=single,
  language=Python,
  showstringspaces=false,
  columns=flexible,
  basicstyle={\small\ttfamily},
  numbers=none,
  breaklines=true,
  breakatwhitespace=true
  tabsize=3
}
\usepackage{amsmath}
\usepackage{amssymb}
\usepackage{amsthm}
\usepackage{harvard}
\usepackage{setspace}
\usepackage{float,color}
\usepackage[pdftex]{graphicx}
\usepackage{hyperref}
\hypersetup{colorlinks,linkcolor=red,urlcolor=blue}
\theoremstyle{definition}
\newtheorem{theorem}{Theorem}
\newtheorem{acknowledgement}[theorem]{Acknowledgement}
\newtheorem{algorithm}[theorem]{Algorithm}
\newtheorem{axiom}[theorem]{Axiom}
\newtheorem{case}[theorem]{Case}
\newtheorem{claim}[theorem]{Claim}
\newtheorem{conclusion}[theorem]{Conclusion}
\newtheorem{condition}[theorem]{Condition}
\newtheorem{conjecture}[theorem]{Conjecture}
\newtheorem{corollary}[theorem]{Corollary}
\newtheorem{criterion}[theorem]{Criterion}
\newtheorem{definition}[theorem]{Definition}
\newtheorem{derivation}{Derivation} % Number derivations on their own
\newtheorem{example}[theorem]{Example}
\newtheorem{exercise}[theorem]{Exercise}
\newtheorem{lemma}[theorem]{Lemma}
\newtheorem{notation}[theorem]{Notation}
\newtheorem{problem}[theorem]{Problem}
\newtheorem{proposition}{Proposition} % Number propositions on their own
\newtheorem{remark}[theorem]{Remark}
\newtheorem{solution}[theorem]{Solution}
\newtheorem{summary}[theorem]{Summary}
%\numberwithin{equation}{section}
\bibliographystyle{aer}
\newcommand\ve{\varepsilon}
\newcommand\boldline{\arrayrulewidth{1pt}\hline}


\begin{document}

\begin{flushleft}
  \textbf{\large{Problem Set \#2}} \\
  OSM \\
  Sophie Sun
\end{flushleft}

\vspace{3mm}
\noindent\textbf{Exercise 1}
\begin{description}
  \item[$\bullet$] \[<x,y> = \frac{1}{4}({\lVert\mathbf{x+y}\rVert}^2 - {\lVert\mathbf{x-y}\rVert}^2)\]
  \[4<x,y> = {\lVert\mathbf{x+y}\rVert}^2 - {\lVert\mathbf{x-y}\rVert}^2\]
  where 
  \[{\lVert\mathbf{x+y}\rVert}^2 = <x+y, x+y> = {\lVert\mathbf{x}\rVert}^2 + {\lVert\mathbf{y}\rVert}^2 + <x,y> + <y,x>\]
  and 
  \[{\lVert\mathbf{x-y}\rVert}^2 = <x-y, x-y> = {\lVert\mathbf{x}\rVert}^2 + {\lVert\mathbf{y}\rVert}^2 - <x,y> - <y,x>\]
    \[{\lVert\mathbf{x+y}\rVert}^2 + {\lVert\mathbf{x-y}\rVert}^2 = {\lVert\mathbf{x}\rVert}^2 + {\lVert\mathbf{y}\rVert}^2 + <x,y> + <y,x> - ({\lVert\mathbf{x}\rVert}^2 + {\lVert\mathbf{y}\rVert}^2 - <x,y> - <y,x>)\]
    \[= <x,y> + <y,x> + <x,y> + <y,x> = 4<x,y>\]
  \item[$\bullet$] \[{\lVert\mathbf{x}\rVert}^2 + {\lVert\mathbf{y}\rVert}^2 = \frac{1}{2}({\lVert\mathbf{x+y}\rVert}^2 + {\lVert\mathbf{x-y}\rVert}^2)\]
  \[2{\lVert\mathbf{x}\rVert}^2 + 2{\lVert\mathbf{y}\rVert}^2 = ({\lVert\mathbf{x+y}\rVert}^2 + {\lVert\mathbf{x-y}\rVert}^2)\]
  where
  \[{\lVert\mathbf{x+y}\rVert}^2 + {\lVert\mathbf{x-y}\rVert}^2=<x+y, x+y> + <x-y, x-y>\]
  \[= {\lVert\mathbf{x}\rVert}^2 + {\lVert\mathbf{y}\rVert}^2 + <x, y> + <y,x> + {\lVert\mathbf{x}\rVert}^2 + {\lVert\mathbf{y}\rVert}^2 - <x, y> - <y, x>\]
  \[=2 {\lVert\mathbf{x}\rVert}^2 + 2{\lVert\mathbf{y}\rVert}^2\]

\end{description}

\noindent\textbf{Exercise 2} 
\begin{description}
\item Using proof from 1, \[Re<x,y> = \frac{1}{4}({\lVert\mathbf{x+y}\rVert}^2 - {\lVert\mathbf{x-y}\rVert}^2)\] 
and 
\[Re<x,iy> = \frac{1}{4}({\lVert\mathbf{x+iy}\rVert}^2 - {\lVert\mathbf{x-iy}\rVert}^2)\] 
Because
\[<x,y> = Re(<x,y>) + iIm(<x,y>)\]
\[<x,y> = \frac{1}{4}({\lVert\mathbf{x+y}\rVert}^2 - {\lVert\mathbf{x-y}\rVert}^2 + i{\lVert\mathbf{x+iy}\rVert}^2 - i{\lVert\mathbf{x-iy}\rVert}^2)\] 
\end{description}

\vspace{3mm}
\noindent\textbf{Exercise 3}
\begin{description}
  \item[$\bullet$] \[cos(\theta) = \frac{<x,y>}{\lVert\mathbf{x}\rVert \lVert\mathbf{y}\rVert}\]
  \[<x, x^5> = \int_0^1 x * x^5 dx =  \int_0^1 x^6 dx = \frac{x^7}{7}|_0^1 = \frac{1}{7}\]
  \[<x,x> = \int_0^1 x * x dx =  \int_0^1 x^2 dx = \frac{x^3}{3}|_0^1 = \frac{1}{3}\]
  \[<x^5, x^5> = \int_0^1 x^5 * x^5 dx =  \int_0^1 x^{10} dx = \frac{x^{11}}{11}|_0^1 = \frac{1}{11}\]
  \[cos(\theta) = \frac{\frac{1}{7}}{\sqrt{\frac{1}{3}} * \sqrt{\frac{1}{11}}}\]
  \[\theta = 34.8^{\circ}\]
  \item[$\bullet$] \[<x^2, x^4> = \int_0^1 x^2 * x^4 dx =  \int_0^1 x^6 dx = \frac{x^7}{7}|_0^1 = \frac{1}{7}\]
    \[<x^2,x^2> = \int_0^1 x^2 * x^2 dx =  \int_0^1 x^4 dx = \frac{x^5}{5}|_0^1 = \frac{1}{5}\]
  \[<x^4, x^4> = \int_0^1 x^4 * x^4 dx =  \int_0^1 x^8 dx = \frac{x^{9}}{9}|_0^1 = \frac{1}{9}\]
  \[cos(\theta) = \frac{\frac{1}{7}}{\sqrt{\frac{1}{5}} * \sqrt{\frac{1}{9}}}\]
  \[\theta = 16.6^{\circ}\]
\end{description}

\vspace{3mm}
\noindent\textbf{Exercise 8}
\begin{description}
  \item[$\bullet$] Sets are orthonormal if the inner product is equal to 0 and norms are equal to 1.
  \[<cos(t), sin(t)> = \frac{1}{\pi} \int_{-\pi}^{\pi} cos(t)sin(t)dt = \frac{1}{\pi} * \frac{sin^2(t)}{2}|_{-\pi}^{\pi} = 0\]
  \[<cos(t), sin(2t)> = \frac{1}{\pi} \int_{-\pi}^{\pi} cos(t)sin(2t)dt = \frac{1}{\pi} * -\frac{sin(2t)sin(t) + 2cos(2t)cos(t)}{3}|_{-\pi}^{\pi} = 0\]
  \[<cos(t), cos(2t)> = \frac{1}{\pi} \int_{-\pi}^{\pi} cos(t)cos(2t)dt = \frac{1}{\pi} * -\frac{cos(2t)sin(t) - 2sin(2t)cos(t)}{3}|_{-\pi}^{\pi} = 0\]
  \[<sin(t), cos(2t)> = \frac{1}{\pi} \int_{-\pi}^{\pi} sin(t)cos(2t)dt = \frac{1}{\pi} * \frac{cos(2t)cos(t) + 2sin(2t)sin(t)}{3}|_{-\pi}^{\pi} = 0\]
  \[<sin(t), sin(2t)> = \frac{1}{\pi} \int_{-\pi}^{\pi} sin(t)sin(2t)dt = \frac{1}{\pi} * \frac{sin(2t)cos(t) - 2cos(2t)sin(t)}{3}|_{-\pi}^{\pi} = 0\]
 \[<cos(2t), sin(2t)> = \frac{1}{\pi} \int_{-\pi}^{\pi} cos(2t)sin(2t)dt = \frac{1}{\pi} * -\frac{cos^2(2x)}{4}|_{-\pi}^{\pi} = 0\]
   \[<cos(t), cos(t)> = \frac{1}{\pi} \int_{-\pi}^{\pi} cos(t)cos(t)dt = \frac{1}{\pi} * (\frac{t}{2} + \frac{sin(2t)}{4}|_{-\pi}^{\pi}) = 1\]
	\[<sin(t), sin(t)> = \frac{1}{\pi} \int_{-\pi}^{\pi} sin(t)sin(t)dt = \frac{1}{\pi} * (\frac{t}{2} - \frac{sin(2t)}{4}|_{-\pi}^{\pi}) = 1\]
  \[<cos(2t), cos(2t)> = \frac{1}{\pi} \int_{-\pi}^{\pi} cos(2t)cos(2t)dt = \frac{1}{\pi} * (\frac{t}{2} + \frac{sin(4t)}{8}|_{-\pi}^{\pi}) = 1\]
  \[<sin(2t), sin(2t)> = \frac{1}{\pi} \int_{-\pi}^{\pi} sin(2t)sin(2t)dt = \frac{1}{\pi} * (\frac{t}{2} - \frac{sin(4t)}{8}|_{-\pi}^{\pi}) = 1\]
  \item[$\bullet$]  $<t, t> = \frac{1}{\pi} \int_{-\pi}^{\pi} t * t dt =\frac{1}{\pi} * \int_{-\pi}^{\pi} t^2 dt = \frac{1}{\pi} * \frac{t^3}{3}|_{-\pi}^{\pi}) = \frac{2\pi^2}{3}$
  \item[$\bullet$] \[proj_X(cos(3t)) = <X, cos(3t)>\frac{X}{\lVert\mathbf{X}\rVert^2}\]
  \[<X, cos(3t)> = \frac{1}{\pi} \int_{-\pi}^{\pi} X * cos(3t) dt = \frac{1}{\pi} * \frac{Xsin(3t)}{3}|_{-\pi}^\pi = 0\]
  \[proj_X(cos(3t)) = 0\]
  \item[$\bullet$] \[proj_X(t) = <X, t>\frac{X}{\lVert\mathbf{X}\rVert^2}\]
  \[<X,t> = \frac{1}{\pi} \int_{-\pi}^{\pi} X * t dt = \frac{1}{\pi} * \frac{Xt^2}{2}|_{-\pi}^\pi = 0\]
  \[proj_X(t) = 0\]
\end{description}

\vspace{3mm}
\noindent\textbf{Exercise 9}
\begin{description}
  \item 
  \[R_{\theta} =
  \begin{bmatrix}
   cos(\theta) & -sin(\theta) \\
   sin(\theta)  & cos(\theta)
\end{bmatrix}
\]
  \[Let \ v \ =
  \begin{bmatrix}
   v_1 \\
   v_2
\end{bmatrix}
\]
  \[Let \ w \ =
  \begin{bmatrix}
   w_1 \\
   w_2
\end{bmatrix}
\]
  \[T(v) =
  \begin{bmatrix}
   cos(\theta)v_1 - sin(\theta)v_2 \\
   sin(\theta)v_1 + cos(\theta)v_2
\end{bmatrix}
\]
  \[T(w) =
  \begin{bmatrix}
   cos(\theta)w_1 - sin(\theta)w_2 \\
   sin(\theta)w_1 + cos(\theta)w_2
\end{bmatrix}
\]
$<T(v), T(w)> = (cos(\theta)v_1 - sin(\theta)v_2)(cos(\theta)w_1 - sin(\theta)w_2) + (sin(\theta)v_1 + cos(\theta)v_2)(sin(\theta)w_1 + cos(\theta)w_2)$
\[= cos^2(\theta)(v_1w_1 + v_2w_2) + sin(\theta)cos(\theta)(-v_1w_2-v_2w_1 + v_1w_2+v_2w_1) + sin^2(\theta)(v_2w_2+v_1w_1)\]
\[= (cos^2(\theta) + sin^2(\theta))(v_1w_1+v_2w_2)\]
\[= v_1w_1 + v_2w_2 = <v,w>\]
\end{description}

\vspace{3mm}
\noindent\textbf{Exercise 10}
\begin{description}
  \item[$\bullet$] Since Q is orthornormal, $<a, b> = <Qa, Qb>$. 
  \[(Qa)^HQb = Q^Ha^HQb = a^Hb\] 
  if and only if $Q^HQ = I$.
  This also works the other way around, since $<Qa, Qb> = <a, b>$.
  \item[$\bullet$] $\lVert\mathbf{(Qx)^2}\rVert = <Qx, Qx> = <x, x> = \lVert\mathbf{x^2}\rVert$ 
  which means that $\lVert\mathbf{Qx}\rVert = \lVert\mathbf{x}\rVert$
  \item[$\bullet$] Because $Q^HQ = I$, $Q^H = Q^{-1},$ 
  \[(Q^H)^H = Q \ and \ (Q^H)^HQ^H = Q^HQ = Q^{-1}Q= I\],
  which means that $Q^{-1}$ is also an orthonormal matrix.
  \item[$\bullet$] Because Q is orthonormal, $(Q^HQ) = I$, which means $(Q^HQ)_{ij} = q_i^Hq_j$.
  \item[$\bullet$] No.
    \[Let \ Q \ =
  \begin{bmatrix}
   2 & 0 \\
   0 & \frac{1}{2}
\end{bmatrix} \]
\[|{det(Q)}| = 1 \  but \ Q^HQ \neq I\]
  \item[$\bullet$] Let Q = $Q_1Q_2$. $Q_1^HQ_1 = I, \ Q_2^HQ_2 = I$
  \[Q = (Q_1Q_2)^H(Q_1Q_2) = Q_1^HQ_2^HQ_1Q_2 = Q_2^HQ_2 = I\] which means that Q is an orthonormal matrix.
\end{description}

\noindent\textbf{Exercise 11} 
\begin{description}
\item When we apply the Gram-Schmidt orthonormalization process to a collection of linearly dependent vectors, $q_k = 0$.
\end{description}

\noindent\textbf{Exercise 16} 
\begin{description}
\item[$\bullet$] Let $A \in \mathbb{M}_{mxn}(\mathbb{F})$ where $rank(A) = n \leq m$
\[\exists Q \in \mathbb{M}_{mxm} \ \textit{and upper triangular} \ R \in \mathbb{M}_{mxn} \ where \ A = QR \]
\[-Q(-Q)^H = QQ^H = I, \ and \ (-Q)^H(-Q) = I \]
\[-R \ \textit{is upper triangular, which means} \ A = QR = (-Q)(-R)\]
\item[$\bullet$] If A is invertible, it has 2 QR decompositions, QR and $\tilde{Q}\tilde{R}$ where R and $\tilde{R}$ are positive. 
\[\tilde{R}^{-1}R = Q^H\tilde{Q}, \ \tilde{R}^{-1}R = I\]
\[R = \tilde{R}, Q = \tilde{Q}\]
\end{description}

\noindent\textbf{Exercise 17} 
\begin{description}
\item \[A^HAx = A^Hb\] 
\[(\hat{Q}\hat{R})^H\hat{Q}\hat{R}x = (\hat{Q}\hat{R})^Hb\] 
\[\hat{R}^H\hat{Q}^H\hat{Q}\hat{R}x = \hat{R}^H\hat{Q}^Hb\]
\[R^{-1}(\hat{R}^H\hat{Q}^H\hat{Q}\hat{R}x = \hat{R}^H\hat{Q}^Hb)\]
\[\hat{R}x = \hat{Q}^Hb\]
\end{description}

\noindent\textbf{Exercise 23} 
\begin{description}
\item \[||x|| = ||x - y + y|| \leq ||x - y|| + ||y||\]
\[||x|| - ||y|| \leq ||x - y||\]
\[||y|| = ||-y|| = ||x - y + x|| \leq ||x - y|| +  ||x||\]
\[||y|| - ||x|| \leq ||x - y|| \]
\end{description}

\noindent\textbf{Exercise 24} 
\begin{description}
\item [$\bullet$] \[\textit{Positivity: because } |f(t)| \geq 0, \int_a^b f(t)dt \geq 0\]
\[\textit{Scale preservation: let } a \in \mathbb{R}. ||af(t)||_{L^1} = \int_a^b |af(t)|dt = |a|\int_a^b f(t)dt\]
$\textit{Triangle inequality: let } g \in C[a,b]. \ because \ |f(t) + g(t)| \leq |f(t)| + |g(t)|, \ \int_a^b |f(t) + g(t)| dt \leq \int_a^b f(t)dt + \int_a^b g(t)dt$
\item [$\bullet$] \[\textit{Positivity: because } |f(t)^2| \geq 0, (\int_a^b f(t)^2dt)^{\frac{1}{2}} \geq 0\]
\[\textit{Scale preservation: let } a \in \mathbb{R}. ||af(t)||_{L^2} = \int_a^b (|af(t)|^2dt)^{\frac{1}{2}} = a\int_a^b f(t)^2dt)^{\frac{1}{2}}\]
$\textit{Triangle inequality: let } g \in C[a,b]. \ because \ |f(t) + g(t)| \leq |f(t)| + |g(t)|, \ (\int_a^b |f(t) + g(t)|^2dt)^{\frac{1}{2}} \leq (\int_a^b f(t)^2dt)^{\frac{1}{2}} + (\int_a^b g(t)^2dt)^{\frac{1}{2}}$
\item [$\bullet$] \[\textit{Positivity: because } |f(x)| \geq 0, \ sup_{x\in[a,b]}|f(x)| \geq 0\]
\[\textit{Scale preservation: let } a \in \mathbb{R}. ||af||_{L^\infty} = sup_{x\in[a,b]}|a||f(x)| = asup_{x\in[a,b]}|f(x)|\]
$\textit{Triangle inequality: let } g \in C[a,b]. \ because \ |f(t) + g(t)| \leq |f(t)| + |g(t)|, \ sup_{x\in[a,b]}|f(x) + g(x)| \leq sup_{x\in[a,b]}|f(x)| + sup_{x\in[a,b]}|g(x)|$
\end{description}

\noindent\textbf{Exercise 26} 
\begin{description}
\item Skipped (1)
\end{description}

\noindent\textbf{Exercise 28} 
\begin{description}
\item [$\bullet$] \[\frac{1}{\sqrt{n}||x||_2} \leq \frac{1}{||x||_1} \leq \frac{1}{||x||_2}\]
\[||Ax||_2 \leq ||Ax||_1 \leq \sqrt{n}||Ax||_2\]
\[\sqrt{n}||A||_2 \geq ||A||_1\]\
\[\frac{1}{\sqrt{n}}||A||_2 \leq ||A||_1 \leq \sqrt{n}||A||_2\]
\item [$\bullet$] \[\frac{1}{\sqrt{n}||x||_\infty} \leq \frac{1}{||x||_2} \leq \frac{1}{||x||_\infty}\]
\[||Ax||_\infty \leq ||Ax||_2 \leq \sqrt{n}||Ax||_\infty\]
\[\sqrt{n}||A||_\infty \geq ||A||_2\]
\[\frac{1}{\sqrt{n}}||A||_\infty \leq ||A||_2 \leq \sqrt{n}||A||_\infty\]
\end{description}

\noindent\textbf{Exercise 29} 
\begin{description}
\item \[||Q|| = sup_{x \neq 0}\frac{||Q_x||_2}{||x||_2} = sup_{x \neq 0}\frac{||x||_2}{||x||_2} = 1\]
\[\textit{Since } ||Ax|| \leq ||A|| ||x||, \ ||Rx|| = sup_{A \neq 0} \frac{||Ax||_2}{||A||} \leq sup_{A \neq 0} \frac{||Ax||_2||x||_2}{||Ax||_2}\]
\[\textit{Since } ||Ax||_2 = ||x||_2 = ||Rx|| \ and \ ||A|| = 1,\]
\[||A|| \leq \frac{||Ax||}{||x||}\]
\end{description}

\noindent\textbf{Exercise 30} 
\begin{description}
\item \[\textit{Positivity: because } ||A||_S \geq 0, ||SAS^{-1}|| \geq 0 \ so \ ||A|| \geq 0\]
\[\textit{Scale preservation: let } a \in \mathbb{R}. ||aA||_S = ||aSAS^{-1}|| = a||SAS^{-1}|| = a||A||_S\]
$\textit{Triangle inequality: let } B \in \mathbb{M}_n(\mathbb{F}). ||A + B||_S = ||S(A + B)S^{-1}|| = ||SAS^{-1} + SBS^{-1}|| \leq ||SAS^{-1}|| + ||SBS^{-1}|| = ||A||_S + ||B||_S$
\end{description}

\noindent\textbf{Exercise 37} 
\begin{description}
\item \[Let \ p \in V, \ p = ax^2 + bx + c = <a, b, c> \]
\[Find \ q \ in V \ \textit{that satisfies } L[p] = p'(1) = p'q	= 2a + b\] 
\[q = (2, 1, 0)\]
\end{description}

\noindent\textbf{Exercise 38} 
\begin{description}
\item \[Let \ p \in V, \ p = ax^2 + bx + c, \ p = (a, b, c)^T, \ p' = D(p) = (0, 2a, b)^T\]
  \[D =
  \begin{bmatrix}
   0 & 0 & 0 \\
   2 & 0 & 0 \\
   0 & 1 & 0
\end{bmatrix}
\]
  \[Hermitian =
  \begin{bmatrix}
   0 & 2 & 0 \\
   0 & 0 &1 \\
   0 & 0 & 0
\end{bmatrix}
\]
\end{description}

\noindent\textbf{Exercise 39} 
\begin{description}
\item [$\bullet$] \[<(S + T)*w, v>_v = <w, (S+T)v>_w\]
\[<w, Sv + Tv>_w = <w, Sv>_w + <w, Tv>_w\]
\[<S*w, v>_v + <T*w, v>_v = <S*w + T*w, v>_v\]
\[(S+T)* = S* + T*\]
\[<(aT)*w, v>_v = <w, (aT)v>_w\]
\[<w, aTv>_w = a<w, Tv>\]
\[a<T*w, v> = <\overline{a}T*w, v>\]
\[(aT)* = \overline{a}T\]
\item [$\bullet$] \[<w, Sv>_w = <S*w, v>_v = \overline{<v, S*w>_v} = \overline{<S**v, w>_w} = <w, S**v>_w\]
\[S = S**\]
\item [$\bullet$] \[<(ST)*v', v>_v = <v', (ST)v>_v = <v', S(Tv)>_v \]
\[<S*v', Tv>_v = <T*S*v', v>_v\]
\[(ST)* = T*S*\]
\item [$\bullet$] \[T*(T*)^{-1} = (TT^{-1})* = I* = I \]
\end{description}

\noindent\textbf{Exercise 40} 
\begin{description}
\item [$\bullet$] \[Let \ B, \ C \in \mathbb{M}_n(\mathbb{F})\] 
\[<B, AC> = <tr(B^HAC)> = tr(A^HB)^HC) = <A^HB, C>\]
\[A* = A^H\]
\item [$\bullet$] \[Let \ A_1, A_2, A_3 \in \mathbb{M}_n(\mathbb{F})\]
\[<A_2, A_3, A_1> = tr(A_2^HA3A1) = tr(A_1A_2^HA_3) = tr((A_2A_1^H)^HA_3) = <A_2, A_1^H, A_3>\]
\[A_1^H = A_1^* \rightarrow <A_2, A_3, A_1> = <A_2A1^*,A_3>\]
\item [$\bullet$] \[Let \ A, B, C \in \mathbb{M}_n(\mathbb{F})\]
\[<B, AC - CA> = <B, AC> - <B, CA>\]
\[<B, CA> = <BA*, C>\]
\[<B, AC> = tr(B^HAC) = tr((A^HB)^HC) = <A^HB, C> = <A*B, C>\]
\[(T_A)^* = T_{A^*}\]
\end{description}

\noindent\textbf{Exercise 44} 
\begin{description}
\item \[Let \ x \in \mathbb{F} \ s.t. \ Ax = b, \ then \ \forall y \in N(A^H),\]
\[<y, b> = <y, Ax> = <A^Hy, x> = <0, x> = 0\]
\[If \ <y, b> \neq 0, \ then \ b \not\in N(A^H) = R(A)\]
\[No \ x \in \mathbb{F}, Ax = b\]
\end{description}

\noindent\textbf{Exercise 45} 
\begin{description}
\item \[Let \ A \in Skew_n(\mathbb{R}), \ B \in Sym_n(\mathbb{R})\]
\[<A, B> = <-A, B> = -<A, B> = tr(AB)\]
\[<A, B> = <B, A> = tr(B^TA) = tr(BA) = tr(AB) \]
\[tr(AB) = -tr(AB) = 0, \ <A, B> = 0\]
\[Let \ B \in Sym_n(\mathbb{R})^\perp\]
\[<B + B^T, B> = tr((B + B^T)B) = tr(BB + B^TB) = tr(BB) + tr(B^TB) = 0\]
\[<B^T, B> = <-B, B>\]
\[B^T = -B\]
\[Sym_n(\mathbb{R})^\perp = Skew_n(\mathbb{R})\]
\end{description}

\noindent\textbf{Exercise 46} 
\begin{description}
\item [$\bullet$] \[A^HAx = A^H(Ax) = 0\]
\[Ax \in N(A^H) \ \rightarrow \ Ax \in R(A) \]
\item [$\bullet$] \[Let \ x \in N(A), \ Ax = 0\]
\[A^HAx = 0, \ x \in N(A^HA)\]
\[Let \ x \in N(A^HA), \ A^HAx = 0\]
\[<Ax, Ax> = x^HA^HAx = 0\]
\[||Ax|| = 0 \ \rightarrow \ Ax = 0\]
\item [$\bullet$]\[n = rank(A) + dimN(A),  \ n = rank(A^HA) + dimN(A^HA)\] 
\[rank(A) = rank(A^HA)\]
\item [$\bullet$] \[n = rank(A) = rank(A^HA) \ \rightarrow \ A^HA \in \mathbb{M}_n\textit{ is nonsingular}\]
\end{description}

\noindent\textbf{Exercise 47} 
\begin{description}
\item [$\bullet$] \[P^2 = A(A^HA)^{-1}A^HA(A^HA)^{-1}A^H = A(A^HA)^{-1}A^H\]
\item [$\bullet$] \[P^H = (A(A^HA)^{-1}A^H)^H = A((A^HA)^{-1})^HA^H = A((A^HA)^H)^{-1}A^H = A(A^HA)^{-1}A^H = P\]
\item [$\bullet$] \[rank(P) = rank(A(A^HA)^{-1}A^H)\]
\[tr(A(A^HA)^{-1}A^H) = tr(A^HA(A^HA)^{-1}) = tr(I)\]
I has dimensions nxn, which means rank(I) = n $\rightarrow$ rank(P) = n
\end{description}

\noindent\textbf{Exercise 48} 
\begin{description}
\item Skipped (2)
\end{description}

\noindent\textbf{Exercise 50} 
\begin{description}
\item \[y^2 = \frac{1}{s} + \frac{rx^2}{s} \ where \ A^HAx = A^Hb \ where\]
  \[A =
  \begin{bmatrix}
   1 & x_1^2 \\
   1 & x_2^2 \\
   \vdots & \vdots \\
   1 & x_n^2
\end{bmatrix}
\]
  \[x =
  \begin{bmatrix}
   \frac{1}{s} \\
   \frac{-r}{s}
\end{bmatrix}
\]
  \[b =
  \begin{bmatrix}
    y_1^2 \\
   y_2^2 \\
   \vdots \\
   y_n^2
\end{bmatrix}
\]
  \[A^Hb =
  \begin{bmatrix}
   \sum_i y_i^2 \\
   \sum_i x_i^2y_i^2
\end{bmatrix}
\]
\end{description}

\end{document}