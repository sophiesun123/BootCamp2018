\documentclass[letterpaper,12pt]{article}
\usepackage{array}
\usepackage{threeparttable}
\usepackage{geometry}
\geometry{letterpaper,tmargin=1in,bmargin=1in,lmargin=1.25in,rmargin=1.25in}
\usepackage{fancyhdr,lastpage}
\pagestyle{fancy}
\lhead{}
\chead{}
\rhead{}
\lfoot{}
\cfoot{}
\rfoot{\footnotesize\textsl{Page \thepage\ of \pageref{LastPage}}}
\renewcommand\headrulewidth{0pt}
\renewcommand\footrulewidth{0pt}
\usepackage[format=hang,font=normalsize,labelfont=bf]{caption}
\usepackage{listings}
\lstset{frame=single,
  language=Python,
  showstringspaces=false,
  columns=flexible,
  basicstyle={\small\ttfamily},
  numbers=none,
  breaklines=true,
  breakatwhitespace=true
  tabsize=3
}
\usepackage{amsmath}
\usepackage{amssymb}
\usepackage{amsthm}
\usepackage{harvard}
\usepackage{setspace}
\usepackage{float,color}
\usepackage[pdftex]{graphicx}
\usepackage{hyperref}
\hypersetup{colorlinks,linkcolor=red,urlcolor=blue}
\theoremstyle{definition}
\newtheorem{theorem}{Theorem}
\newtheorem{acknowledgement}[theorem]{Acknowledgement}
\newtheorem{algorithm}[theorem]{Algorithm}
\newtheorem{axiom}[theorem]{Axiom}
\newtheorem{case}[theorem]{Case}
\newtheorem{claim}[theorem]{Claim}
\newtheorem{conclusion}[theorem]{Conclusion}
\newtheorem{condition}[theorem]{Condition}
\newtheorem{conjecture}[theorem]{Conjecture}
\newtheorem{corollary}[theorem]{Corollary}
\newtheorem{criterion}[theorem]{Criterion}
\newtheorem{definition}[theorem]{Definition}
\newtheorem{derivation}{Derivation} % Number derivations on their own
\newtheorem{example}[theorem]{Example}
\newtheorem{exercise}[theorem]{Exercise}
\newtheorem{lemma}[theorem]{Lemma}
\newtheorem{notation}[theorem]{Notation}
\newtheorem{problem}[theorem]{Problem}
\newtheorem{proposition}{Proposition} % Number propositions on their own
\newtheorem{remark}[theorem]{Remark}
\newtheorem{solution}[theorem]{Solution}
\newtheorem{summary}[theorem]{Summary}
%\numberwithin{equation}{section}
\bibliographystyle{aer}
\newcommand\ve{\varepsilon}
\newcommand\boldline{\arrayrulewidth{1pt}\hline}


\begin{document}

\begin{flushleft}
  \textbf{\large{Problem Set \#1}} \\
  OSM \\
  Sophie Sun
\end{flushleft}

\vspace{3mm}
\noindent\textbf{Exercise 1.3}
\begin{description}
  \item[$\bullet$] Not an algebra
  \item[$\bullet$] Algebra
  \item[$\bullet$] Algebra and $\sigma$-algebra
\end{description}

\noindent\textbf{Exercise 1.7} 
\begin{description}
\item \{$\emptyset$, X\} is the smallest possible $\sigma$-algebra because by definition, a $\sigma$-algebra must have an empty set, which is included in this, and its complement must also be in the set, which is also included in \{$\emptyset$, X\}. On the other hand, the power set is the largest possible $\sigma$-algebra because it lists basically all the combinations of X.
\end{description}

\vspace{3mm}
\noindent\textbf{Exercise 1.10}
\begin{description}
  \item[$\bullet$] Since each $S_\alpha$ is a $\sigma$-algebra, that means that $\emptyset \in S_\alpha$ for all $\alpha$, which implies that $\emptyset \in \cap_\alpha S_\alpha$
  \item[$\bullet$] If $A \in S_\alpha$ for all $\alpha$, then $A^c \in S_\alpha$ for all $\alpha$ because each $S_\alpha$ is a $\sigma$-algebra, which implies that $A^c \in \cap S_\alpha$
  \item[$\bullet$] If $A \in S_\alpha$ for all $\alpha$, then $\cup A \in S_\alpha$ for all $\alpha$ because each $S_\alpha$ is a $\sigma$-algebra, which implies that $\cup A \in \cap_\alpha S_\alpha$
  \item[$\bullet$] Therefore, if \{$S_\alpha$\} is a $\sigma$-algebra, then $\cap_\alpha S_\alpha$ is also a $\sigma$-algebra.
\end{description}

\vspace{3mm}
\noindent\textbf{Exercise 1.17}
\begin{description}
  \item[$\bullet$] Because $\mu$ is a nonnegative function and \[\mu(\cup_{i=1}^\infty){A_i} =
  \sum_{i=1}^\infty \mu{A_i}\] then \[B = A \cup (B \cap A^c) \  because \  A \subset B \] then \[\mu (A) + \mu (B \cap A^c) = \mu (B) \] then \[\mu(A) \leq \mu(B) \] then \[ A, B \in S, A \subset B \ then \  \mu(A) \leq \mu(B)\]
  \item[$\bullet$] Let $A_m$, $A_n$ $\in$ A.
  
  Let D = $A_m \cap A_n^c$, E = $A_n \cap A_m^c$, and F = $A_m \cap A_n$.
  
  Because \[\mu(\cup_{i=1}^\infty){A_i} = \sum_{i=1}^\infty \mu{A_i}\]
  
  then $\mu(A_n \cup A_m) = \mu(D \cup E \cup F)$, and because D, E, and F are disjoint, then \[\mu(A_n \cup A_m) = \mu(D) + \mu(E) + \mu(F)\]. Solving for the left side of the equation: \[\mu(A_n) + \mu(A_m) = \mu(D \cup F) + \mu(E \cup F)\]
\[\mu(A_n) + \mu(A_m) = \mu(D) + \mu(F) + \mu(E) + \mu(F)\] which means
\[\mu(D) + \mu(E) + \mu(F) \leq \mu(D) + \mu(F) + \mu(E) + \mu(F)\] which means
  if \[\{A_i\}_{i=1}^\infty \subset A \ , then \ \mu(\cup_{i=1}^\infty A_i) \leq \sum_{i=1}^\infty \mu(A_i)\]
\end{description}

\vspace{3mm}
\noindent\textbf{Exercise 1.18}
\begin{description}
  \item[$\bullet$] Because $B \in S$ then $(A \cap B) \in S$. Because $\lambda(A) = \mu(A \cap B)$ and $B \in S$, then $\lambda(A) = \mu(A \cap B)$ is also a measure $(X,S)$.
  
\end{description}

\vspace{3mm}
\noindent\textbf{Exercise 1.20}
\begin{description}
  \item[$\bullet$] Let $B_1 = A_1$ and $B_i = A_i \cap A_{i-1}^c$ for $i \leq 2$. Also, $A = \cup_{n \in N} B_n$ and $A_n = \cup_{n = 1}^n B_n$. This means that $\lim_{n\to\infty} (A_1 \cap A_n^c) = A_1 \cap A^c $ 
  
  Using the proof from i), \[\mu(A_1) - \mu(A_n) = \lim_{n\to\infty} \mu(A_1 \cap A_n^c) = \mu(A_1 \cap A^c) = \mu(A_1) = \mu(A_n)\]
  which means 
  \[\lim_{n\to\infty} \mu(A_n) = \mu(\cap_{i=1}^\infty A_i)\]
\end{description}

\noindent\textbf{Exercise 2.10} 
\begin{description}
\item Since both E $\in$ X and B $\in$ X, we know that both are in X. Therefore, there are three options: E = B, $E \cap B = \emptyset$ or $E \cap B \neq \emptyset$.

If E = B, then $\mu^*(B) = \mu^*(B \cap E) + \mu^*(B \cap E^c)$ where $\mu^*(B \cap E) = \mu^*(B)$ and $\mu^*(B \cap E^c)$ = 0, which means $\mu^*(B) = \mu^*(B)$, which means $\mu^*(B) = \mu^*(B \cap E) + \mu^*(B \cap E^c)$.

If $E \cap B = \emptyset$, then $\mu^*(B) = \mu^*(B \cap E) + \mu^*(B \cap E^c)$ where $\mu^*(B \cap E)$ = 0 and $\mu^*(B \cap E^c) = \mu^*(B)$, which means $\mu^*(B) = \mu^*(B)$, which means $\mu^*(B) = \mu^*(B \cap E) + \mu^*(B \cap E^c)$.

If $E \cap B \neq \emptyset$, then $\mu^*(B) = \mu^*(B \cap E) + \mu^*(B \cap E^c)$ where $\mu^*(B \cap E) + \mu^*(B \cap E^c) = \mu^*(B)$, which means $\mu^*(B) = \mu^*(B)$, which means $\mu^*(B) = \mu^*(B \cap E) + \mu^*(B \cap E^c)$.

This means that \[\mu^*(B) = \mu^*(B \cap E) + \mu^*(B \cap E^c)\]
\end{description}

\noindent\textbf{Exercise 2.14} 
\begin{description}
\item Let O be the set of open sets and let A be defined as $O \subset A$. Because $B(\mathbb{R}) = \sigma(O) \subset \sigma(A) \subset \{S_\alpha\}$
\end{description}

\noindent\textbf{Exercise 3.1} 
\begin{description}
\item Let $\{x_i\}_{i=1}^\infty$ be elements of X. For any $\epsilon > 0$, $A_i = (x_i - \frac{\epsilon}{2^i}, x_i + \frac{\epsilon}{2^i})$, then $X \subset \cup_{i=1}^\infty A_i$ and $M(\cup A_i) \leq \sum_{i=1}^\infty (2^{1-i}\epsilon) = 2\epsilon$, which means $M(X) = 0$.
\end{description}

\noindent\textbf{Exercise 4.15} 
\begin{description}
\item Because $\{s : 0 \leq s \leq f \}$ and $\{s : 0 \leq s \leq g \}$ where s is simple, measurable, $\int_E f d\mu = \int_E g d\mu$
\end{description}

\noindent\textbf{Note:} 
\begin{description}
\item Sorry I just don't understand much of this math at all, even after working with other people in the program and watching multiple YouTube videos on measure theory. I especially don't understand how to prove things and what certain equations mean - I've never taken a proof-based math class. I genuinely feel bad for not turning in a complete problem set, but I really don't understand this and feel like this math is beyond my level - is there any advice you could give me in terms of how to catch up to this level? 
\end{description}

\end{document}